\documentclass{article}
\usepackage{shortvrb,verbatim}
\MakeShortVerb|
\title{\texttt{reverxii}\\
  Playing Reversi in \TeX{}}
\author{Bruno Le Floch}
\date{September 20, 2015}
\newcommand{\cs}[1]{\texttt{\char`\\ #1}}
\newcommand{\meta}[1]{$\langle \mathit{#1}\rangle$}
\newcommand{\enquote}[1]{``#1''}
\begin{document}
\maketitle

\begin{abstract}
  \texttt{reverxii.tex} is a $938$ character long \TeX{} program
  which lets you play Reversi against your favorite typestting engine.
  To play, run plain \TeX{}, on the command line, in dvi-mode,
  on the file \texttt{reverxii.tex}. In most distributions,
  this should mean running \texttt{tex reverxii.tex} in a terminal.
  To typeset the documentation, run \LaTeX{} on \texttt{reverxii.dtx},
  with any engine (\emph{e.g.}, \texttt{pdflatex reverxii.dtx}).
\end{abstract}

\section{The code}

Line breaks can be removed, and the stripped down code takes $938$
characters.
{\small \verbatiminput{reverxii.tex}}

To play a two-player game, change |1=P| to |0=P| near the end:
this changes the computer player from $0$ to $1$, hence disabling it.

\section{Explanation}

\subsection{Some comments}

The name \texttt{reverxii.tex} is of course a reference to the infamous
\texttt{xii.tex}, also on CTAN, which typesets the lyrics of the
\textit{Twelve days of Christmas} using code that very few can understand.
In my case, I aimed for the shortest possible code, rather than the
most obfuscated. In particular, I did not assign weird catcodes
other than making most characters active.

Since I was aiming for short code, the text presented to the player
is concise, but hopefully enough to leave the game understandable
and usable. Despite aiming for short code, I thought it would be
neat to typeset a record of the game as it goes: after all, \TeX{}
is a typesetting program. This used up around $38$ characters, putting
me above the $900$ character line.

One technique used to make the code shorter is to rename any primitive
used more than once into a single active character. Also, counters
holding the information about the board are accessed directly by number.

A careful reader would notice that changing one of the remaining
one-character control sequences to active characters (|$| is still %$
unused) would gain one character. I didn't do it, because it messes
up the code-highlighting of my editor |:)|. Of course, I chose the
control sequences which are not active characters to be those appearing
the least, only twice each.

\subsubsection{On with the code!}

First set up the page dimensions. This would not be enough for pdf output.
\begin{verbatim}
\vsize5cm
\hsize3cm
\end{verbatim}

Since plain \TeX{} does not provide the \cs{typeout} command, we are
using \cs{message}, hence need to setup a new line character. It is
arbitrarily chosen to be |*|, which will be made active and \cs{let}
to \cs{cr}.
\begin{verbatim}
\newlinechar`*
\end{verbatim}

Then a first loop to make many characters active. The loop ends when
reaching the trailing brace group: there we make spaces active, then
redefine |~| for the next loop. We still have an annoying |13~| in
the input stream. Introduce a short-hand for count registers whose
number starts with |1|, then remove |13| by setting |\count11| to
|213| (this counter is used for allocation later on). The next loop
is triggered by the |~| which we left.
\begin{verbatim}
\def~#1{\catcode`#113~}
~IJKLMNOPQRSTUVWXYZjqz@.[|](/+^'";:?,_)!*
{ 13\def~#1#2{\let#1#2~}\def+{\count1}+1=2}
\end{verbatim}

At this stage, |~| is defined to take two arguments, \cs{let} the
first to the second, and repeat. As announced |*| becomes \cs{cr},
so that it will be a new line both in messages and in alignments.
We try to keep a relatively consistent naming scheme: opening
conditionals are left delimiters, \cs{or} and \cs{else} are
middle delimiters, and \cs{fi} is a right bracket. Other primitives
which are used a lot are also given short names. The loop ends by
redefining |~| to \cs{def}.
\begin{verbatim}
*\cr
[\ifnum
(\ifcase
|\or
/\else
]\fi
N\number
@\advance
X\expandafter
!\global
?\message
~\def
\end{verbatim}

It is time to allocate counters. Unfortunately, \cs{newcount}
is \cs{outer}, so it is unpractical to use. We have set |+1|,
\emph{aka.} |\count11| to $213$, and will allocate counters
starting from that number upwards. Repeatedly advance |+1|
by $1$ and define the next character to be the counter number |+1|,
then repeat. As always, the loop is interrupted by making it
redefine the looping macro |.| to be a counter. We use the trailing
dot to assign it the value $-1$, then assign a couple of counters:
starting player, other player, and the initial position:
the squares in rows and columns $4$ and $5$ are initially
filled in a cross pattern.
\begin{verbatim}
.#1{@+1 1\countdef#1+1.}.IJKLPQRSTUVWYZ.-1P1T2+44P+55P+45T+54T
\end{verbatim}
Let us summarize which counters are used where:
\begin{itemize}
\item [P] is the current player ($1$ for |-| or $2$ for |0|);
\item [T] is the other player;
\item [Q] is the row number;
\item [J] is the column number;
\item [S] is the step size in the row direction;
\item [K] is the step size in the column direction;
\item [R] is the score difference, positive when $0$ is winning;
\item [V] is used when computing the value of placing a piece at
  the position (|Q|,|J|);
\item [W] is the value of the best possible move according to the AI,
  also used to end the game if neither player can move;
\item [U] is the row number of the best move;
\item [L] is the column number of the best move;
\item [Y] is a boolean, true ($0$) most of the time,
  it has to do with when we flip or not, but I don't
  understand it now, help welcome;
\item [Z] is a temporary counter, used locally to compute
  how much a given cell matters (\emph{i.e.} corners are important),
  and used globally as a boolean to indicate whether to flip pieces
  or not.
\end{itemize}

The board is stored in counters |1|\meta{row}\meta{column}.
An empty square is has the value $0$, a square for player |-|
has value $1$, and player |0| corresponds to the value $2$.
The macro |j| retrieves that value, assuming that the row and
column numbers are stored as |Q| and |J|, and returns |.|,
that is, $-1$, if outside the board. Recall that |[| is \cs{ifnum},
|/| is \cs{else}, and |]| is \cs{fi}. When |Q| and |J| are
within bounds, the value is retrieved by |^| as
|\count1|\meta{row}\meta{column}. Note that |j| and |^| are
only safe to use on the left-hand side of an \cs{ifnum} test
(because of expansion issues), and that |^| can be used
on the left-hand side of an assignment.
\begin{verbatim}
~j{[0<Q[9>Q[0<J[9>J^/.]/.]/.]/.]}
~^{+NQNJ}
\end{verbatim}

We often need to loop over numbers from $1$ to $8$; here is a macro.
\begin{verbatim}
~:#1{#11#12#13#14#15#16#17#18}
\end{verbatim}

The macros to print the board, both to the dvi and to the console.
|M| spews its argument as a \cs{message} (|?|), and directly typeset.
This is used rather directly to print the first and last lines (|'|),
which are simply alignment cells containing each number from $1$ to $8$,
with some care given to spaces and new lines.
The |_| macro receives a digit and the corresponding capital letter
as arguments, and outputs that row of the board. First place the
letter on the left of the board, then loop from $1$ to $8$,
typesetting and \cs{message}ing a space, |-| or |0|, depending on
the value of the relevant count register. Note the two spaces
in the definition: the first ends the counter's number, the second
is typeset (in plain \TeX{}, active spaces expand to a normal space).
\begin{verbatim}
~M#1{?{#1}#1}
~'{M :{&M}&M{*}}
~_#1#2{M#2:{\B#1}&M#2&M{*}}
~\B#1#2{&M{(+#1#2  |-|0]}}
\end{verbatim}

The input is done by |q|, which queries the user until they give
a correct input, so that |Q| and |J| are in the range $[1,8]$.
Prompt the user with a small \cs{message}, giving an example
of what move he could do (only true at the start, but the hope
is that the player will understand what the input format is).
The code that follows is similar to \LaTeXe{}'s \cs{@onelevel@sanitize}.
We extract the two first characters from the the \cs{meaning}
of the user's input (remember, |X| is \cs{expandafter}),
as |#2| and |#3| of \cs{D}. Grab the character code of each,
relative to the characters |@| (row) or |0| (column).
The closing parenthesis is where most of the work is done.
It sets |V| to the value of placing a piece in the cell
$(|Q|,|J|)$, zero if the move does not flip any of the opponent's
pieces, or if the cell is outside the board. After performing
that calculation, if |V| is zero, the move was not valid,
and we query the user again.
\begin{verbatim}
~q{?{Row and column? e.g. E6*}\read.to\EX\D\meaning\E  ;}
~\D#1->#2#3#4;{Q`#2@Q-`@J`#3@J-`0)(V?{Invalid move #2#3.}q]}
\end{verbatim}

So\ldots{} how do we compute the \enquote{value} of a move?
It is automatically invalid if |j| does not return $0$:
either the cell is occupied, or it is outside the board.
Then for each of the $8$ directions around the cell,
we count the number of pieces that are flipped in that
direction. The direction is stored as two counters,
|S| and |K|, each $\pm 1$ or $0$. We call |,| a first
time to test whether flipping should happen in that direction,
and, if it is (as indicated by the global value of |Z|),
call it again to actually flip. The call to |,| must happen
within a group, because it directly changes the row and column
numbers |Q| and |J|, following which cell is being queried.
\begin{verbatim}
~){V0 (jS1z1z0z.S0z1z.S.z1z0z.]}
~z#1{{K#1Z1{Y1,}(Z,]}}
\end{verbatim}

The macro |,| is recursive. At each step, move $(|Q|,|J|)$
in the direction $(|S|,|K|)$. Then, if that cell (|j|)
contains a piece belonging to the other player (|T|),
do some stuff |(Y!^P!;2]O| and repeat. What is it that
we do? Well, if the condition |Y| is met (I don't remember
how that works), we set the current cell to belong to the
player, globally, and change the score difference by $2$
(see |;|), also globally. Then, we compute with |O| the value
corresponding to the cell that we might be flipping (see |O|).

Otherwise (|/|), if the cell (|j|) contains a piece from the
current player (|P|), it means we have reached the end of a
run of the form \meta{initial cell} \meta{opponent's pieces}
\meta{own piece}, hence the \meta{opponent's pieces} should
count as flipped if we place our piece in the \meta{initial
  cell}. Until there, all changes to |V| were local, returning
to the old value at the end of the group that |,| is contained
in. Since the run in that direction was successful, we escape
the value of |V| from the group with |\global V=V|. Under
some conditions, we set the boolean |Z| to true, globally (|!Z0|),
which triggers a second call to |,| with different setting,
and actually flips the opponent's pieces.
\begin{verbatim}
~,{@QS@JK[j=T(Y!^P!;2]O,/[j=P!VV(I(Y/!Z0]]]]}
\end{verbatim}

I moved those |O| and |"| guys a little bit in this explanation,
but not in the original implemenation, because it is hard
to sync. We assign weights to each of the $64$ cells:
\begin{center}
  \halign{&\hfil#\hfil\cr
    9 & 1 & 6 & 6 & 6 & 6 & 1 & 9 \cr
    1 & 1 & 2 & 2 & 2 & 2 & 1 & 1 \cr
    6 & 2 & 4 & 4 & 4 & 4 & 2 & 6 \cr
    6 & 2 & 4 & 4 & 4 & 4 & 2 & 6 \cr
    6 & 2 & 4 & 4 & 4 & 4 & 2 & 6 \cr
    6 & 2 & 4 & 4 & 4 & 4 & 2 & 6 \cr
    1 & 1 & 2 & 2 & 2 & 2 & 1 & 1 \cr
    9 & 1 & 6 & 6 & 6 & 6 & 1 & 9 \cr}
\end{center}
All weights are positive, so that every move which flips a piece
ends up with a positive overall value. The AI would be better if
the places next to corners had a negative weight, but I would have
too much code to rewrite for that to work. We really have three
types of rows and three types of columns. Converting from |Q|
or |J| is done by |"|, then we assemble the two as a number
in the range $[0,8]$, and use another \cs{ifcase} construction
to produce the weights.
\begin{verbatim}
~O{Z"Q\multiplyZ3@Z"J@V(Z9|1|6|1|1|2|6|2|4] }
~"#1{(#1|0|1|2|2|2|2|1|0]}
\end{verbatim}

The counter |R| keeps track of the score difference,
and is updated with |;2| (when flipping a piece) or |;1| (when
adding a piece). The counter |R| should be \cs{advance}d (|@|)
by a positive amount when the current player |P| is player |0|,
and a negative amount for player |-|.
\begin{verbatim}
~;{@R(P|-]}
\end{verbatim}

After printing the board, we go through every cell and find the
one with the highest value. The macro \cs{A}, does one row, hence
|\:\A| does all the rows. Store the argument as the row number |Q|,
then loop over columns. After setting the column number |J| to
its argument, \cs{C} calls |)|, which as explained above computes
the value of placing a piece there, throws away that case if it flips
nothing, otherwise also counts the weight of the current cell. Then
update the best value |W| and the best row |U| and column |L| if
the new |V| is larger than |W|.
\begin{verbatim}
~\A#1{Q#1:\C}
~\C#1{J#1)[0<VO[V>WWVUQLJ]]}
\end{verbatim}

We won't need |~| as \cs{def} anymore, so we reuse it as the main command.
\begin{itemize}
\item First exchange the players: set |P| equal to |T|, but first expand
  the value of |P| after |T| to set that as well.
\item Secondly, open an alignment, with a repeating preamble adding a space
  at its end. Then message a new-line (we should be using |?| rather than
  |M| here, I think) to keep a clean output. Afterwards, print the top line
  with |'|, the eight lines of the bulk with |_|, and the bottom line,
  which happens to end with |?{*}*|, \emph{i.e.}, ends with \cs{cr}:
  we can thus close the alignment, and cause \TeX{} to output the page.
\item After printing, it is time to check whether there is a move or not.
  We don't want to flip pieces at this stage, hence set the boolean |I|
  to false ($1$). And we reset the best value to $0$, unless it was
  already $0$ (which means that the previous player had no available
  move), in which case we set it to $-1$. Then loop over rows, finding
  the best value (see \cs{A}). Reset the boolean |I| to be true.
\item If a move was found in the previous step ($|W|>0$), either
  use it if the player is the AI, or query the user. The various
  booleans are set up in such a way as to do the flipping, so
  calling |)| does it. Then also put a player's piece in the current
  cell |^|, and increase the score difference by $1$.
\item Finally, if neither player could move, declare the game ended,
  give the score, and \cs{end} the run. Otherwise, repeat.
\end{itemize}
Of course, after defining |~|, we call it. Let's play!
\begin{verbatim}
~~{
  PXTXTNP
  \halign{&## *M{*}'_1A_2B_3C_4D_5E_6F_7G_8H'}\vfil\break
  I1 W(W./0] :\AI0
  [0<W
    [1=PQUJL/q]
    )
    ^P;1
  ]
  [.=W
    ?{(RTie/ Player [0>R-/0] wins by N[0>R-]R].}
    X\end
  ]
  ~
}
~
\end{verbatim}

\end{document}